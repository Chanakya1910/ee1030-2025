%iffalse%\let\negmedspace\undefined
\let\negthickspace\undefined
\documentclass[journal,12pt,onecolumn]{IEEEtran}
\usepackage{cite}
\usepackage{amsmath,amssymb,amsfonts,amsthm}
\usepackage{algorithmic}
\usepackage{graphicx}
\usepackage{textcomp}
\usepackage{xcolor}
\usepackage{caption}
\usepackage{txfonts}
\usepackage{listings}
\usepackage{enumitem}
\usepackage{mathtools}
\usepackage{gensymb}
\usepackage{comment}
\usepackage[breaklinks=true]{hyperref}
\usepackage{tkz-euclide} 
\usepackage{listings}
\usepackage{gvv}                                        
%\def\inputGnumericTable{}                                 
\usepackage[latin1]{inputenc}   
\usepackage{xparse}
\usepackage{color}                                            
\usepackage{array}                                            
\usepackage{longtable}                                       
\usepackage{calc}                                             
\usepackage{multirow}
\usepackage{multicol}
\usepackage{hhline}                                           
\usepackage{ifthen}                                           
\usepackage{lscape}
\usepackage{tabularx}
\usepackage{array}
\usepackage{float}
\newtheorem{theorem}{Theorem}[section]
\newtheorem{problem}{Problem}
\newtheorem{proposition}{Proposition}[section]
\newtheorem{lemma}{Lemma}[section]
\newtheorem{corollary}[theorem]{Corollary}
\newtheorem{example}{Example}[section]
\newtheorem{definition}[problem]{Definition}
\newcommand{\BEQA}{\begin{eqnarray}}
\newcommand{\EEQA}{\end{eqnarray}}
\usepackage{float}
%\newcommand{\define}{\stackrel{\triangle}{=}}
\theoremstyle{remark}
\usepackage{ circuitikz }
%\newtheorem{rem}{Remark}
% Marks the beginning of the document this one
\title{ AE : AEROSPACE ENGINEERING}
\author{EE25BTECH11018-Darisy Sreetej}
\begin{document}
\maketitle

\textbf{Q.1 -- Q.25 carry one mark each.}

\begin{enumerate}

\item The constraint $A^2 = A$ on any square matrix $A$ is satisfied for
\begin{enumerate}
\begin{multicols}{2}
\item the identity matrix only
\item the null matrix only
\item both the identity matrix and the null matrix
\item no square matrix $A$
\end{multicols}
\end{enumerate}
\hfill(GATE AE 2012)


\item The general solution of the differential equation  
$\dfrac{d^2y}{dt^2} + \dfrac{dy}{dt} - 2y = 0$ is
\begin{enumerate}
\begin{multicols}{2}
\item $Ae^{-t} + Be^{2t}$
\item $Ae^{-2t} + Be^{-t}$
\item $Ae^{-2t} + Be^{t}$
\item $Ae^{t} + Be^{2t}$
\end{multicols}
\end{enumerate}
\hfill(GATE AE 2012)



\item An aircraft in trimmed condition has zero pitching moment at
\begin{enumerate}
\begin{multicols}{2}
\item its aerodynamic centre
\item its centre of gravity
\item $25\%$ of its mean aerodynamic chord
\item $50\%$ of its wing root chord
\end{multicols}
\end{enumerate}
\hfill(GATE AE 2012)



\item In an aircraft, constant roll rate can be produced using ailerons by applying
\begin{enumerate}
\begin{multicols}{2}
\item a step input
\item a ramp input
\item a sinusoidal input
\item an impulse input
\end{multicols}
\end{enumerate}
\hfill(GATE AE 2012) 



\item For a symmetric airfoil, the lift coefficient for zero degree angle of attack is
\begin{enumerate}
\begin{multicols}{4}
\item $-1.0$
\item $0.0$
\item $0.5$
\item $1.0$
\end{multicols}
\end{enumerate}
\hfill(GATE AE 2012) 



\item The critical Mach number of an airfoil is attained when
\begin{enumerate}
\item the freestream Mach number is sonic
\item the freestream Mach number is supersonic
\item the Mach number somewhere on the airfoil is unity
\item the Mach number everywhere on the airfoil is supersonic
\end{enumerate}
\hfill(GATE AE 2012) 



\item The shadowgraph flow visualization technique depends on
\begin{enumerate}
\item the variation of the value of density in the flow
\item the first derivative of density with respect to spatial coordinate
\item the second derivative of density with respect to spatial coordinate
\item the third derivative of density with respect to spatial coordinate
\end{enumerate}
\hfill(GATE AE 2012)



\item The Hohmann ellipse used as earth-Mars transfer orbit has
\begin{enumerate}
\begin{multicols}{2}
\item apogee at earth and perigee at Mars
\item both apogee and perigee at earth
\item apogee at Mars and perigee at earth
\item both apogee and perigee at Mars
\end{multicols}
\end{enumerate}
\hfill(GATE AE 2012)



\item The governing equation for the static transverse deflection of a beam under an uniformly distributed load, according to Euler-Bernoulli (engineering) beam theory, is a
\begin{enumerate}
\item $2^{\text{nd}}$ order linear homogenous partial differential equation
\item $4^{\text{th}}$ order linear non-homogenous ordinary differential equation
\item $2^{\text{nd}}$ order linear non-homogenous ordinary differential equation
\item $4^{\text{th}}$ order nonlinear homogenous ordinary differential equation
\end{enumerate}
\hfill(GATE AE 2012)



\item The Poisson's ratio, $\nu$ of most aircraft grade metallic alloys has values in the range
\begin{enumerate}
\begin{multicols}{4}
\item $-1 \leq \nu \leq 0$
\item $0 \leq \nu \leq 0.2$
\item $0.2 \leq \nu \leq 0.4$
\item $0.4 \leq \nu \leq 0.5$
\end{multicols}
\end{enumerate}
\hfill(GATE AE 2012) 



\item The value of $k$ for which the system of equations $x+2y+kz=1$ ; $2x+ky+8z=3$ has no solution is
\begin{enumerate}
\begin{multicols}{4}
\item $0$
\item $2$
\item $4$
\item $8$
\end{multicols}
\end{enumerate}
\hfill(GATE AE 2012)



\item If $u(t)$ is a unit step function, the solution of the differential equation $m\dfrac{d^2x}{dt^2}+kx=u(t)$ in Laplace domain is
\begin{enumerate}
\begin{multicols}{2}
\item $\dfrac{1}{s(ms^2+k)}$
\item $\dfrac{1}{ms^2+k}$
\item $\dfrac{s}{ms^2+k}$
\item $\dfrac{1}{s^2(ms^2+k)}$
\end{multicols}
\end{enumerate}
\hfill(GATE AE 2012) 



\item The general solution of the differential equation $\dfrac{dy}{dx}-2\sqrt{y}=0$ is
\begin{enumerate}
\begin{multicols}{2}
\item $y-\sqrt{x}+C=0$
\item $y-x+C=0$
\item $\sqrt{y}-\sqrt{x}+C=0$
\item $\sqrt{y}-x+C=0$
\end{multicols}
\end{enumerate}
\hfill(GATE AE 2012)



\item During the ground roll manoeuvre of an aircraft, the force(s) acting on it parallel to the direction of motion
\begin{enumerate}
\begin{multicols}{2}
\item is thrust alone.
\item is drag alone.
\item are both thrust and drag.
\item are thrust, drag and a part of both weight and lift.
\end{multicols}
\end{enumerate}
\hfill(GATE AE 2012)



\item An aircraft in a steady climb suddenly experiences a $10\%$ drop in thrust. After a new equilibrium is reached at the same speed, the new rate of climb is
\begin{enumerate}
\begin{multicols}{2}
\item lower by exactly $10\%$.
\item lower by more than $10\%$.
\item lower by less than $10\%$.
\item an unpredictable quantity.
\end{multicols}
\end{enumerate}
\hfill(GATE AE 2012)



\item In an aircraft, the dive manoeuvre can be initiated by
\begin{enumerate}
\begin{multicols}{2}
\item reducing the engine thrust alone.
\item reducing the angle of attack alone.
\item generating a nose down pitch rate.
\item increasing the engine thrust alone.
\end{multicols}
\end{enumerate}
\hfill(GATE AE 2012)



\item In an aircraft, elevator control effectiveness determines
\begin{enumerate}
\begin{multicols}{2}
\item turn radius.
\item rate of climb.
\item forward-most location of the centre of gravity.
\item aft-most location of the centre of gravity.
\end{multicols}
\end{enumerate}
\hfill(GATE AE 2012)



\item The Mach angle for a flow at Mach $2.0$ is
\begin{enumerate}
\begin{multicols}{4}
\item $30^{\degree}$
\item $45^{\degree}$
\item $60^{\degree}$
\item $90^{\degree}$
\end{multicols}
\end{enumerate}
\hfill(GATE AE 2012) 



\item For a wing of aspect ratio $AR$, having an elliptical lift distribution, the induced drag coefficient is (where $C_L$ is the lift coefficient)
\begin{enumerate}
\begin{multicols}{4}
\item $\dfrac{C_L}{\pi AR}$
\item $\dfrac{C_L^2}{\pi AR}$
\item $\dfrac{C_L}{2\pi AR}$
\item $\dfrac{C_L^2}{\pi AR^2}$
\end{multicols}
\end{enumerate}
\hfill(GATE AE 2012) 



\item Bernoulli's equation is valid under steady state 
\begin{enumerate}
\item only along a streamline in inviscid flow, and between any two points in potential flow.
\item  between any two points in both inviscid flow and potential flow.
\item between any two points in inviscid flow, and only along a streamline in potential flow.
\item  only along a streamline in both inviscid flow and potential flow. 
\end{enumerate}
\hfill(GATE AE 2012)



\item The ratio of flight speed to the exhaust velocity for maximum propulsion efficiency is
\begin{enumerate}
\begin{multicols}{4}
\item $0.0$
\item $0.5$
\item $1.0$
\item $2.0$
\end{multicols}
\end{enumerate}
\hfill(GATE AE 2012)



\item The ideal static pressure coefficient of a diffuser with an area ratio of $2.0$ is
\begin{enumerate}
\begin{multicols}{4}
\item $0.25$
\item $0.50$
\item $0.75$
\item $1.0$
\end{multicols}
\end{enumerate}
\hfill(GATE AE 2012)



\item A rocket is to be launched from the bottom of a very deep crater on Mars for earth return. The specific impulse of the rocket, measured in seconds, is to be normalized by the acceleration due to gravity at
\begin{enumerate}
\begin{multicols}{2}
\item the bottom of the crater on Mars.
\item Mars standard ``sea level''.
\item earth's standard sea level.
\item the same depth of the crater on earth.
\end{multicols}
\end{enumerate}
\hfill(GATE AE 2012)



\item In a semi-monocoque construction of an aircraft wing, the skin and spar webs are the primary carriers of
\begin{enumerate}
\item shear stresses due to an aerodynamic moment component alone.
\item normal (bending) stresses due to aerodynamic forces.
\item shear stresses due to aerodynamic forces alone.
\item shear stresses due to aerodynamic forces and a moment component.
\end{enumerate}
\hfill(GATE AE 2012)



\item The logarithmic decrement measured for a viscously damped single degree of freedom system is $0.125$. The value of the damping factor in \% is closest to
\begin{enumerate}
\begin{multicols}{4}
\item $0.5$
\item $1.0$
\item $1.5$
\item $2.0$
\end{multicols}
\end{enumerate}
\hfill(GATE AE 2012)



\item The integration $\int_{0}^{1} x^3 dx$ computed using trapezoidal rule with $n = 4$ intervals is \dots.
\hfill(GATE AE 2012)



\item An aircraft has a steady rate of climb of $300 \, m/s$ at sea level and $150 \, m/s$ at $2500 \, m$ altitude. The time taken (in sec) for this aircraft to climb from $500 \, m$ altitude to $3000 \, m$ altitude is \dots.
\hfill(GATE AE 2012)



\item An airfoil generates a lift of $80 \, N$ when operating in a freestream flow of $60 \, m/s$. If the ambient pressure and temperature are $100 \, kPa$ and $290 \, K$ respectively (specific gas constant is $287 \, J/kg \cdot K$), the circulation on the airfoil in $m^2/s$ is \dots.
\hfill(GATE AE 2012)



\item A rocket motor has combustion chamber temperature of $2600 \, K$ and the products have molecular weight of $25 \, g/mol$ and ratio of specific heats $1.2$. The universal gas constant is $8314 \, J/kg \cdot mol \cdot K$. The value of theoretical $c^*$ (in $m/s$) is \dots.
\hfill(GATE AE 2012)



\item The mode shapes of an un-damped two degrees of freedom system are 
$\{1 \;\; 0.5\}^T$ and $\{1 \;\; -0.675\}^T$. The corresponding natural frequencies are 
$0.45 \, Hz$ and $1.247 \, Hz$. The maximum amplitude (in mm) of vibration of the first 
degree of freedom due to an initial displacement of $\{2 \;\; 1\}^T$ (in mm) and zero 
initial velocities is \dots
\hfill(GATE AE 2012)



\textbf{Questions Q.31 to Q.55 are multiple choice type}



\item The $n^{\text{th}}$ derivative of the function $y = \frac{1}{x+3}$ is
\begin{enumerate}
\begin{multicols}{4}
\item $\frac{(-1)^n n!}{(x+3)^{n+1}}$
\item $\frac{(-1)^{n+1} n!}{(x+3)^{n+1}}$
\item $\frac{(-1)^n (n+1)!}{(x+3)^n}$
\item $\frac{(-1)^n n!}{(x+3)^n}$
\end{multicols}
\end{enumerate}
\hfill(GATE AE 2012)



\item The volume of a solid generated by rotating the region between semi-circle 
$y = 1 - \sqrt{1-x^2}$ and straight line $y = 1$, about x-axis, is
\begin{enumerate}
\begin{multicols}{4}
\item $\pi^2 - \frac{4}{3} \pi$
\item $4\pi^2 - \frac{1}{3} \pi$
\item $\pi^2 - \frac{3}{4} \pi$
\item $\frac{\pi}{4} \pi^2 - \pi$
\end{multicols}
\end{enumerate}
\hfill(GATE AE 2012)



\item One eigenvalue of the matrix $A = \myvec{2 & 7 & 10  5 & 2 & 25  1 & 6 & 5}$ is $-9.33$. One of the other eigenvalues is
\begin{enumerate}
\begin{multicols}{4}
\item $18.33$
\item $-18.33$
\item $18.33 - 9.33i$
\item $18.33 + 9.33i$
\end{multicols}
\end{enumerate}
\hfill(GATE AE 2012)



\item If an aircraft takes off with $10\%$ less fuel in comparison to its standard configuration, its range is
\begin{enumerate}
\begin{multicols}{2}
\item lower by exactly $10\%$
\item lower by more than $10\%$
\item lower by less than $10\%$
\item an unpredictable quantity
\end{multicols}
\end{enumerate}
\hfill(GATE AE 2012)



\item An aircraft has an approach speed of $144$ kmph with a descent angle of $6.6^\degree$. If the aircraft load factor is $1.2$ and constant deceleration at touch down is $0.25g$ $(g = 9.81 \, \text{m/s}^2)$, its total landing distance approximately over a $15$ m high obstacle is
\begin{enumerate}
\begin{multicols}{4}
\item $1830$ m
\item $1380$ m
\item $830$ m
\item $380$ m
\end{multicols}
\end{enumerate}
\hfill(GATE AE 2012)



\item An aircraft is trimmed straight and level at true air speed (TAS) of $100$ m/s at standard sea level (SSL). Further, pull of $5$ N holds the speed at $90$ m/s without re-trimming at SSL (air density = $1.22$ kg/m$^3$). To fly at $3000$ m altitude (air density = $0.91$ kg/m$^3$) and $120$ m/s TAS without re-trimming, the aircraft needs
\begin{enumerate}
\begin{multicols}{2}
\item $1.95$ N upward force
\item $1.95$ N downward force
\item $1.85$ N upward force
\item $1.75$ N downward force
\end{multicols}
\end{enumerate}
\hfill(GATE AE 2012)



\item An oblique shock wave with a wave angle $\beta$ is generated from a wedge angle of $\theta$. The ratio of the Mach number downstream of the shock to its normal component is
\begin{enumerate}
\begin{multicols}{4}
\item $\sin(\beta - \theta)$
\item $\cos(\beta - \theta)$
\item $\sin(\theta - \beta)$
\item $\cos(\theta - \beta)$
\end{multicols}
\end{enumerate}
\hfill(GATE AE 2012)



\item In a closed-circuit supersonic wind tunnel, the convergent-divergent (C-D) nozzle and test section are followed by a C-D diffuser to swallow the starting shock. Here, we should have the
\begin{enumerate}
\item diffuser throat larger than the nozzle throat and the shock located just at the diffuser throat.
\item diffuser throat larger than the nozzle throat and the shock located downstream of the diffuser throat.
\item diffuser throat of the same size as the nozzle throat and the shock located just at the diffuser throat.
\item diffuser throat of the same size as the nozzle throat and the shock located downstream of the diffuser throat.
\end{enumerate}
\hfill(GATE AE 2012)



\item A vortex flowmeter works on the principle that the Strouhal number of 0.2 is a constant over a wide range of flow rates. If the bluff-body diameter in the flowmeter is $20$ mm and the piezo-electric transducer registers the vortex shedding frequency to be $10$ Hz, then the velocity of the flow would be measured as
\begin{enumerate}
\begin{multicols}{4}
\item $0.1$ m/s
\item $1$ m/s
\item $10$ m/s
\item $100$ m/s
\end{multicols}
\end{enumerate}
\hfill(GATE AE 2012)



\item The stagnation temperatures at the inlet and exit of a combustion chamber are $600$ K and $1200$ K, respectively. If the heating value of the fuel is $44$ MJ/kg and specific heat at constant pressure for air and hot gases are $1.005$ kJ/kg.K and $1.147$ kJ/kg.K respectively, the fuel-to-air ratio is
\begin{enumerate}
\begin{multicols}{4}
\item $0.0018$
\item $0.018$
\item $0.18$
\item $1.18$
\end{multicols}
\end{enumerate}
\hfill(GATE AE 2012)



\item A solid propellant of density $1800$ kg/m$^3$ has a burning rate law $r = 6.65 \times 10^{-3} p^{0.45}$ mm/s, where $p$ is pressure in Pascals. It is used in a rocket motor with a tubular grain with an initial burning area of $0.314$ m$^2$. The characteristic velocity is $1450$ m/s. What should be the nozzle throat diameter to achieve an equilibrium chamber pressure of $50$ bar at the end of the ignition transient?
\begin{enumerate}
\begin{multicols}{4}
\item $35$ mm
\item $38$ mm
\item $41$ mm
\item $45$ mm
\end{multicols}
\end{enumerate}
\hfill(GATE AE 2012)



\item A bipropellant liquid rocket motor operates at a chamber pressure of $40$ bar with a nozzle throat diameter of $50$ mm. The characteristic velocity is $1540$ m/s. If the fuel-oxidizer ratio of the propellant is $1.8$, and the fuel density is $900$ kg/m$^3$, what should be the minimum fuel tank volume for a burn time of $8$ minutes
\begin{enumerate}
\begin{multicols}{4}
\item $1.65$ m$^3$
\item $1.75$ m$^3$
\item $1.85$ m$^3$
\item $1.95$ m$^3$
\end{multicols}
\end{enumerate}
\hfill(GATE AE 2012)



\item The propellant in a single stage sounding rocket occupies $60\%$ of its initial mass. If all of it is expended instantaneously at an equivalent exhaust velocity of $3000$ m/s, what would be the altitude attained by the payload when launched vertically?
(Neglect drag and assume acceleration due to gravity to be constant at $9.81$ m/s$^2$)
\begin{enumerate}
\begin{multicols}{4}
\item $315$ km
\item $335$ km
\item $365$ km
\item $385$ km
\end{multicols}
\end{enumerate}
\hfill(GATE AE 2012)



\item The Airy stress function, $\phi = \alpha x^2 + \beta xy + \gamma y^2$ for a thin square panel of size $l \times l$ automatically satisfies compatibility. If the panel is subjected to uniform tensile stress, $\sigma_o$, on all four edges, the traction boundary conditions are satisfied by
\begin{enumerate}
\begin{multicols}{2}
\item $\alpha = \sigma_o / 2; \ \beta = 0; \ \gamma = \sigma_o / 2$
\item $\alpha = \sigma_o; \ \beta = 0; \ \gamma = \sigma_o$
\item $\alpha = 0; \ \beta = \sigma_o / 4; \ \gamma = 0$
\item $\alpha = 0; \ \beta = \sigma_o / 2; \ \gamma = 0$
\end{multicols}
\end{enumerate}
\hfill(GATE AE 2012)



\item The boundary condition of a rod under longitudinal vibration is changed from fixed-fixed to fixed-free. The fundamental natural frequency of the rod is now $k$ times the original frequency, where $k$ is
\begin{enumerate}
\begin{multicols}{4}
\item $\frac{1}{2}$
\item $2$
\item $\frac{1}{\sqrt{2}}$
\item $\sqrt{2}$
\end{multicols}
\end{enumerate}
\hfill(GATE AE 2012)



\item A spring-mass system is viscously damped with a viscous damping constant $c$. The energy dissipated per cycle when the system is undergoing a harmonic vibration $X \cos\omega_d t$ is given by
\begin{enumerate}
\begin{multicols}{4}
\item $\pi c \omega_d X^2$
\item $\pi \omega_d X^2$
\item $\pi c \omega_d^2 X$
\item $\pi c \omega_d^2 X^2$
\end{multicols}
\end{enumerate}
\hfill(GATE AE 2012)



\item Buckling of the fuselage skin can be delayed by 
\begin{enumerate}
\begin{multicols}{2}
\item increasing internal pressure
\item  placing stiffeners farther apart
\item reducing skin thickness
\item  placing stiffeners farther and decreasing internal pressure
\end{multicols}
\end{enumerate}
\hfill(GATE AE 2012)



\textbf{Common Data Questions}

\textbf{Common Data for Questions 48 and 49:}
A wing and tail are geometrically similar, while tail area is one-third of the wing area and distance between two aerodynamic centres is equal to wing semi-span ($b/2$). In addition, following data is applicable: 
$\epsilon_a = 0.3, \ C_{L} = 1.0, \ C_{L_\epsilon} = 0.08 \ \text{/ deg.}, \ \overline{c} = 2.5 \ \text{m}, \ b = 30 \ \text{m}, \ C_{M_{ac}} = 0, \ \eta_t = 1$. The symbols have their usual aerodynamic interpretation.

\item The maximum distance that the centre of gravity can be behind aerodynamic centre without destabilizing the wing-tail combination is
\begin{enumerate}
\begin{multicols}{4}
\item $0.4$ m
\item $1.4$ m
\item $2.4$ m
\item $3.4$ m
\end{multicols}
\end{enumerate}
\hfill(GATE AE 2012)



\item The angle of incidence of tail to trim the wing-tail combination for a $5\%$ static margin is
\begin{enumerate}
\begin{multicols}{4}
\item $-1.4^\degree$
\item $-0.4^\degree$
\item $0.4^\degree$
\item $1.4^\degree$
\end{multicols}
\end{enumerate}
\hfill(GATE AE 2012)



\textbf{Common Data for Questions 50 and 51:}
A thin long circular pipe of $10$ mm diameter has porous walls and spins at $60$ rpm about its own axis. Fluid is pumped out of the pipe such that it emerges radially relative to the pipe surface at a velocity of $1$ m/s. [Neglect the effect of gravity.]

\item What is the radial component of the fluid's velocity at a radial location $0.5$ m from the pipe axis?
\begin{enumerate}
\begin{multicols}{4}
\item $0.01$ m/s
\item $0.1$ m/s
\item $1$ m/s
\item $10$ m/s
\end{multicols}
\end{enumerate}
\hfill(GATE AE 2012)



\item What is the tangential component of the fluid's velocity at the same radial location as above?
\begin{enumerate}
\begin{multicols}{4}
\item $0.01$ m/s
\item $0.03$ m/s
\item $0.10$ m/s
\item $0.31$ m/s
\end{multicols}
\end{enumerate}
\hfill(GATE AE 2012)



\textbf{Linked Answer Questions}
\textbf{Statement for Linked Answer Questions 52 and 53:}
Air at a stagnation temperature of $15^\degree$C and stagnation pressure $100$ kPa enters an axial compressor with an absolute velocity of $120$ m/s. Inlet guide vanes direct this absolute velocity to the rotor inlet at an angle of $18^\degree$ to the axial direction. The rotor turning angle is $27^\degree$ and the mean blade speed is $200$ m/s. The axial velocity is assumed constant through the stage.

\item The blade angle at the inlet of the rotor is
\begin{enumerate}
\begin{multicols}{4}
\item $25.5^\degree$
\item $38.5^\degree$
\item $48.5^\degree$
\item $59.5^\degree$
\end{multicols}
\end{enumerate}
\hfill(GATE AE 2012)



\item If the mass flow rate is $1$ kg/s, the power required to drive the compressor is
\begin{enumerate}
\begin{multicols}{4}
\item $50.5$ kW
\item $40.5$ kW
\item $30.5$ kW
\item $20.5$ kW
\end{multicols}
\end{enumerate}
\hfill(GATE AE 2012)



\textbf{Statement for Linked Answer Questions 54 and 55:}
A thin-walled spherical vessel ($1$ m inner diameter and $10$ mm wall thickness) is made of a material with $|\sigma_y| = 500$ MPa in both tension and compression.

\item The internal pressure $p_t$ at yield, based on the von Mises yield criterion, if the vessel is floating in space, is approximately
\begin{enumerate}
\begin{multicols}{4}
\item $500$ MPa
\item $250$ MPa
\item $100$ MPa
\item $20$ MPa
\end{multicols}
\end{enumerate}
\hfill(GATE AE 2012)



\item If the vessel is evacuated (internal pressure = 0) and subjected to external pressure, yielding according to the von Mises yield criterion (assuming elastic stability until yield)
\begin{enumerate}
\begin{multicols}{2}
\item occurs at about half the pressure Y p 
\item occurs at about double the pressure Y p
\item occurs at about the same pressure Y p 
\item  never occurs. 
\end{multicols}
\end{enumerate}
\hfill(GATE AE 2012)



\textbf{General Aptitude (GA) Questions}

\textbf{Q.56 -- Q.60 carry one mark each.}

\item Choose the most appropriate alternative from the options given below to complete the following sentence:
I \dots to have bought a diamond ring.
\begin{enumerate}
\begin{multicols}{4}
\item have a liking
\item should have liked
\item would like
\item may like
\end{multicols}
\end{enumerate}
\hfill(GATE AE 2012)



\item Choose the most appropriate alternative from the options given below to complete the following sentence:
Food prices \dots again this month.
\begin{enumerate}
\begin{multicols}{4}
\item have raised
\item have been raising
\item have been rising
\item have arose
\end{multicols}
\end{enumerate}
\hfill(GATE AE 2012)



\item Choose the most appropriate alternative from the options given below to complete the following sentence:
\textit{The administrators went on to implement yet another unreasonable measure, arguing that the measures were already \dots and one more would hardly make a difference.}
\begin{enumerate}
\begin{multicols}{4}
\item reflective
\item utopian
\item luxuriant
\item unpopular
\end{multicols}
\end{enumerate}
\hfill(GATE AE 2012)



\item Choose the most appropriate alternative from the options given below to complete the following sentence:
\textit{To those of us who had always thought him timid, his \dots came as a surprise.}
\begin{enumerate}
\begin{multicols}{4}
\item intrepidity
\item inevitability
\item inability
\item inertness
\end{multicols}
\end{enumerate}
\hfill(GATE AE 2012)



\item The arithmetic mean of five different natural numbers is $12$. The largest possible value among the numbers is
\begin{enumerate}
\begin{multicols}{4}
\item $12$
\item $40$
\item $50$
\item $60$
\end{multicols}
\end{enumerate}
\hfill(GATE AE 2012)



\textbf{Q.61 -- Q.65 carry two marks each.}

\item Two policemen, A and B, fire once each at the same time at an escaping convict. The probability that A hits the convict is three times the probability that B hits the convict. If the probability of the convict not getting injured is $0.5$, the probability that B hits the convict is
\begin{enumerate}
\begin{multicols}{4}
\item $0.14$
\item $0.22$
\item $0.33$
\item $0.40$
\end{multicols}
\end{enumerate}
\hfill(GATE AE 2012)



\item The total runs scored by four cricketers P, Q, R, and S in years 2009 and 2010 are given in the following table: 
\begin{tabular}[12pt]{ |c| c| } 
    \hline
    {Group I: Aircraft mode} & {Group II: Property}\\ 
    \hline
    P: Short period mode & 1: Coupled roll-yaw oscillations\\
    \hline 
    Q: Wing rock & 2: Angle of attack remains constant \\
    \hline
    R: Phugoid mode & 3: Roll oscillations \\
    \hline   
    S: Dutch roll & 4: Speed remains constant\\
    \hline
\end{tabular}

The player with the lowest percentage increase in total runs is 
\begin{enumerate}
\begin{multicols}{4}
    \item P
    \item Q
    \item R
    \item S
    \end{multicols}
\end{enumerate}
\hfill(GATE AE 2012)



\item If a prime number on division by 4 gives a remainder of 1, then that number can be expressed as  
\begin{enumerate}
\begin{multicols}{2}
        \item sum of squares of two natural numbers
        \item sum of cubes of two natural numbers
        \item sum of square roots of two natural numbers
        \item sum of cube roots of two natural numbers
        \end{multicols}
    \end{enumerate}
    \hfill(GATE AE 2012)


    
    \item Two points (4, $p$) and (0, $q$) lie on a straight line having a slope of $3/4$. The value of $(p - q)$ is  
    \begin{enumerate}
    \begin{multicols}{4}
        \item -3
        \item 0
        \item 3
        \item 4
        \end{multicols}
    \end{enumerate}
    \hfill(GATE AE 2012)


    
    \item \textbf{In the early nineteenth century, theories of social evolution were inspired less by Biology than by the conviction of social scientists that there was a growing improvement in social institutions. Progress was taken for granted and social scientists attempted to discover its laws and phases.}  
    
    Which one of the following inferences may be drawn with the greatest accuracy from the above passage?  
    
    Social scientists  
    \begin{enumerate}
    \begin{multicols}{2}
        \item did not question that progress was a fact.
        \item did not approve of Biology.
        \item framed the laws of progress.
        \item emphasized Biology over Social Sciences.
        \end{multicols}
    \end{enumerate}
    \hfill(GATE AE 2012)



\newpage

\begin{center}
\begin{tabular}[12pt]{|l|l|}
\hline 
  P.Processes   & 1.Characteristics / Applications \\ \hline
  Q.Gas Metal Arc Welding    & 2.Joining of thick plates \\ \hline
  R.Tungsten Inert Gas Welding &  3.Consumable electrode wire  \\ \hline
  S.Electroslag Welding & 4.Joining of cylindrical dissimilar materials \\ \hline
\end{tabular}
\end{center}
    
\end{enumerate}
\end{document}